\newpage
\section{Analisi complessiva}

\subsection{Struttura}
Come si può evincere dal resto dell'analisi, la struttura del sito è piuttosto confusionaria, in primis la Homepage. Lo stile non è unico e uniforme, e questo può far spazientire l'utente ma, soprattutto, rischia di confondere le idee e compromettere lo \textbf{schema mentale} che l'utente si fa alla prima vista della pagina. \\
La visualizzazione delle news in Homepage, per quanto utile e, in parte, funzionale, non possiede una struttura lineare, e può perciò essere di difficile interpretazione a un occhio distratto. L'alternarsi di immagini piccole e grandi e di anteprime a griglia e a lista trasmette inoltre una sensazione di poca cura dei particolari e poca attenzione alla leggibilità del sito. \\
Un simile discorso si può applicare anche alle pagine interne: la scelta fatta è stata quella di privilegiare le immagini, ponendo una grande immagine di copertina all'inizio di ogni articolo; questo inficia sui timer degli utenti, i quali dedicheranno poi meno tempo a leggere il contenuto a causa del tempo sprecato nello scroll della pagina.

\subsection{Pubblicità}
Il sito possiede due slot per l'inserimento di pubblicità, per il momento ancora vuoti (pubblicizzano infatti lo slot stesso).

\vspace{30pt}
\begin{figure}[htbp]
\begin{center}
\includegraphics[width=35em]{img/pubblicita1}
\caption{Pubblicità in alto}
\end{center}
\end{figure}
\vspace{30pt}

\begin{figure}[htbp]
\begin{center}
\includegraphics[width=10em]{img/pubblicita2}
\caption{Pubblicità a lato dell'articolo}
\end{center}
\end{figure}
\vspace{30pt}

Le posizioni sono rispettivamente in alto in centro (affianco al logo) e sul lato destro degli articoli. Si può dire poco a proposito poiché vuoti, ma le posizioni sono comunque ottimali: le pubblicità poste in alto sono le seconde più gradite, e quelle a destra sono terze. Entrambe le posizioni, inoltre, non disturbano la lettura della pagina poiché poco invadenti.

\subsection{404}
Il sito prevede una pagina di 404, molto utile a tranquillizzare l'utente nel caso in cui manchi una risorsa cercata.

\vspace{30pt}
\begin{figure}[htbp]
\begin{center}
\includegraphics[width=35em]{img/404}
\caption{La pagina d'errore}
\end{center}
\end{figure}
\vspace{30pt}

La pagina non è particolarmente curata, ma svolge il suo lavoro; propone infatti all'utente più scappatoie dalla pagina, poiché si può tornare alla Homepage oppure saltare direttamente a uno degli ultimi sei articoli pubblicati.